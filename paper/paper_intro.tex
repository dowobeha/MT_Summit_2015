% !TEX root = paper.tex
% !TEX encoding = UTF-8 Unicode

\section{Introduction}

Post-editing is the process whereby a human user corrects the output of a machine translation system.
%
The use of basic post-editing tools by bilingual human translators has been shown to yield substantial increases in terms of productivity \citep{2010_PBML_Plitt_Masselot} as well as improvements in translation quality \citep{2013_CHI_Green_etal} when compared to bilingual human translators working without assistance from machine translation and post-editing tools.
%
More sophisticated interactive interfaces \citep{2000_NAACL_Langlais_etal,2009_CL_Barrachina,2009_ACL_Koehn,2012_AMTA_Denkowski_Lavie} may also provide benefit \citep{2009_MT_Koehn}.

Small-scale studies have suggested that monolingual human post-editors, working without knowledge of the source language, can also improve the quality of machine translation output \citep{2005_NIST_CallisonBurch,2010_NAACL_Koehn,2013_WPTP_Mitchell_etal}, especially if well-designed tools provide automated linguistic analysis of source sentences \citep{2009_EACL_Albrecht_etal}.
%
\citet{2014_WMT_Schwartz_etal} confirmed this result with eight monolingual post-editors on a larger 3000 sentence test corpus.
%
\citet{2014_WPTP_Schwartz} demonstrated that a monolingual post-editor who is a domain expert in the translated material can successfully post-edit.

In this work, we conduct a bilingual post-editing experiment (\S\ref{sec:methodology}), in which six Russian-English bilingual translation students were asked to post-edit news articles that were previously machine translated from Russian into English.
%
We find (\S\ref{sec:results}) that post-editing quality is consistently higher, by a statistically significant amount, when bilingual post-editors are presented with alignment data.
