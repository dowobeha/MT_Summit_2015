% !TEX root = paper.tex
% !TEX encoding = UTF-8 Unicode

\section{Introduction}

Post-editing is the process whereby a human user corrects the output of a machine translation system.
%
The use of basic post-editing tools by bilingual human translators has been shown to yield substantial increases in terms of productivity \citep{2010_PBML_Plitt_Masselot} as well as improvements in translation quality \citep{2013_CHI_Green_etal} when compared to bilingual human translators working without assistance from machine translation and post-editing tools.
%
More sophisticated interactive interfaces \citep{2000_NAACL_Langlais_etal,2009_CL_Barrachina,2009_ACL_Koehn,2012_AMTA_Denkowski_Lavie} may also provide benefit \citep{2009_MT_Koehn}.







The question of how a post-editing interface should be configured and presented to users is a fundamentally interdisciplinary and empirical one.
%
Issues of user interface design, human factors, translation studies, and machine translation quality are all likely relevant.
%
Phrase-based machine translation can be configured to produce alignment data that indicates which machine translated target language words correspond to which original source language words.
%
In most prior work that examined the efficacy of post-editing machine translation, post-editors were presented with machine translations (and in most cases the original source language sentences) without also being presented with source-to-target alignment links.

This work begins an attempt to answer two novel questions regarding post-editing interface design: 
%
To what extent, if at all, does the presentation of source-to-target word-level alignment links affect the quality or speed of post-editing?
%
Is any such effect, if it exists, dependent on certain aspects of machine translation quality, or on the language pair?



To address these questions, we conduct a bilingual post-editing experiment (\S\ref{sec:methodology}) where bilingual post-editors are presented with machine translation output of varying quality, with and without word-level alignment link visualization.
%
In the first condition, we ask six Russian-English bilingual translation students to post-edit two Russian language news articles starting with relatively low quality English machine translation.
%
In the second condition, we ask four Spanish-English bilingual translation students to post-edit two Spanish language news articles starting with relatively high quality English machine translation.
%
We find (\S\ref{sec:results}) that when machine translation quality is low, post-editing quality is consistently higher, by a statistically significant amount, when bilingual post-editors are presented with alignment data.
%
We find no statistically significant effect when machine translation quality is high.
%
We also found that for both Russian-English and Spanish-English the mean post-editing times were shorter for texts with alignment than for texts without alignment. These differences were not significant, but the difference for the Russian-English texts approached significance. 
%
Finally, in \S\ref{sec:background} we briefly survey the current state of post-editing research and situate this work within the context of related work in post-editing.